{\color{gray}\hrule}
\section{Connection Lost (0,0)}
{\color{gray}\hrule}

\subsection{Why (0,0)?}
The most frequently placed pixel is in the upper left (0,0). With this in mind, I hypothesized that the corners were easy to reach and recognizable places to reach in the 2000 x 2000 grid. I tested this by checking the activity in the other corners. Since the canvas expanded twice, the other corners were only available after some expansions. This would explain why the upper left corner is at the top since it was always accessible. Trying the other corners, I found that the corners were generally hot spots. The bottom right, (1999, 1999), had 31437 pixels, bottom left, (1999, 0), 30882 pixels, and the top right, (0, 1999), 22763 pixels. This confirmed that users valued corners as locations to place their pixels.

\begin{figure}[H]
\centering
\includegraphics{visuals/0_0_100_78_mode_1_1}
    \caption{Most recent common pixel placements in the top left corner at the 78th hour}
\end{figure}

\subsection{Why This Corner?}

Since corners are considered valuable spaces, what was created here? In the top left, users created the "Connection lost" sign from RuneScape which also appeared in the top left. This sign was created by r/2007scape community which maintained this image in r/place 2017 and continued this into r/place 2022. Furthermore, this iconic message appeared in the top left of RuneScape, so it was created in the same respective place. With the top left corner as the most available corner and r/2007scape determined to maintain this image, the conflict focused user activity onto (0, 0).

\begin{figure}[H]
\centering
\includegraphics[width=0.5\textwidth]{visuals/connection_lost}
    \caption{Clean "Connection Lost" Image}
    \href{https://www.reddit.com/r/place/comments/tukrrl/the_connection_lost_banner_what_is_this_anyway/}{r/place figtan}
\end{figure}
